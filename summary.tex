\chapter{Summary} \label{ch:summary}

This thesis presented the \SB, an interactive tool for exploring system dependence graphs (SDGs). First we described 
the structure of the SDG and how it represents certain programming constructs, such as function calls and methods, as 
well as the additions and simplifications applied by the tool. The tool was created with four concrete use cases in 
mind, which we discussed by means of examples. The next part of the thesis was written as a user's manual for the \SB 
and the related IEC project infrastructure. The manual details the steps for creating and building a project, using the 
\SB for concrete use cases, and explains additional features such as dead code detection. Lastly, we discussed the 
implementation, with a particular focus on details that may be important to someone trying to understand the \SB code 
base and extend it.

The \SB was developed in the course of our cooperation with \href{http://www.keba.com}{Keba AG} at the \emph{Christian 
Doppler Laboratory on Monitoring and Evolution of Very-Large-Scale Software Systems} (CDL MEVSS for short). During his 
PhD work, Florian Angerer created our implementation of the SDG. The goal of the \SB is to make it possible to 
interactively explore that SDG and use it for program comprehension and evolution, and for solving problems.

The SDG is built based on the AST of an IEC program. First the PDGs are built which encode control and data 
dependencies within a single procedure. Then PDGs are instantiated and connected to form the object-sensitive (i.e.\ 
instance-based) SDG. The SDG contains nodes for statements, calls, procedures, and parameters; it contains edges for 
control and for data dependencies. The \SB also introduces container nodes and dummy nodes.
  
There are four concrete use cases identified by Keba developers. The \emph{Executions} use case may be used to identify 
how a statements may be executed, by exploring control dependencies of a statement or procedure. The \emph{Variable   
Assignment} use case shows all statements which modify a particular variable, and for those statements shows their 
Executions. The \emph{Change Impact} use case follows both data and control dependencies starting from any node in the 
SDG. It can be used to determine which statements or variables are influenced at all by the target node. Finally the 
\emph{Change Cause} use case is an inverse change impact, it can find variables or statements which influence the 
execution of a particular statement or the value of a variable.

The \SB as an Eclipse plugin uses the build infrastructure provided by the IEC editor plugin. The \SB operations can be 
accessed from the IEC editor context menu, navigation to the code from the \SB is also easily accessible. The \SB 
provides convenient features such as marking and hiding nodes that have been examined. Dead code detection based on a 
hardware configuration is possible, dead code is then highlighted in the editor and the \SB.

The \SB is implemented as an Eclipse plugin in a model-view-controller pattern. The model represents a use case and 
builds a so called display graph based on the user interactions. The display graph is an abstraction of the SDG and 
AST, which allows the view to be implemented independently of the underlying SDG type. An \SB view only needs to 
implement an interface to be able to work with the controller. There is one concrete implementation for the Eclipse 
workbench, which uses Draw2D for rendering graph objects. The \SB may be controlled from within Eclipse, but can also 
accept commands trough an interface for external programs, which uses named pipes for communication.

\section{Experiences}

Our first approach to the \SB used the \href{https://d3js.org/}{D3.js} framework in a browser embedded into the Eclipse 
IDE\footnotemark. D3.js was chosen because of it's ability to quickly create appealing visualizations, combined with 
built-in support for transitions. However, we soon discovered that this would lead to significant code duplication. 
Some of the supporting data structures would need to be implemented both on the Eclipse side in Java and on the 
browser side in JavaScript. Communication between both sides also needed to happen through JSON-serialized objects, 
with was another unnecessary burden. We therefore decided to implement the view in pure Java using the Draw2D 
framework, which is native to Eclipse.

\footnotetext{We used \href{https://www.teamdev.com/jxbrowser}{JxBrowser} by Ukrainian company TeamDev. The library 
provides a Chromium browser and works flawlessly when integrated into an SWT part. It is well documented and the 
developers react quickly to bug reports.}

We also discovered that providing a stable layout for an incrementally built graph is rather difficult. Our layout 
algorithm of choice (KLay) handles this case more or less gracefully, however there are still significant layout 
changes at times. More research is required to provide a stable layout for  changing graphs to support a nice user 
experience.
