\chapter{Introduction}

A \emph{system dependence graph} (SDG), introduced by Horwitz et al.\ \cite{DBLP:journals/toplas/HorwitzRB90}, encodes 
all control and data dependencies in a whole program. The SDG represents procedures by their \emph{program dependence 
graphs} (PDGs). A PDG consists of a node for the procedure itself, a number of formal parameter nodes for parameters 
and global variables used in the procedure, and nodes for every statement in the procedure. The nodes are connected by 
edges representing the control and data dependencies inside the procedure. The SDG also represents the dependencies 
between procedures in the whole program, based on one or more entry points into the program. Further, the SDG is 
instance-based, which means in object-oriented terms, that a method called on different instances of a class will be 
represented by distinct instances of the method's PDG.

In configurable software systems that need to meet a wide range of customer requirements, variability is of great 
importance. Understanding variable software is essential for its evolution and there are several approaches for dealing 
with the analysis of variable software systems. One important application of program analysis is \emph{change impact 
analysis} (CIA), the identification of the potential consequences of a change. An approach for change impact analysis 
of configurable software was developed by Angerer et al.\ \cite{DBLP:conf/kbse/AngererGPG15}. This approach uses a 
system dependence graph to represent the whole program. An extension of that is the \emph{conditional system dependence 
graph} (CSDG) introduced by Angerer \cite{DBLP:conf/splc/AngererPLGG14}, which also represents variability information.

System dependence graphs are constructed based on the source code of a whole program through static program analysis. 
This means it is not necessary to run a program for analyzing it, but it also means that the analysis cannot use any 
information which might only be available at runtime.

In the concrete case of building an SDG, the major downside of using static analysis is that information about object 
instances needs to be extracted from the source code alone, which might not be easy or at all possible. For example, in 
languages like Java that support dynamic allocation of objects, complicated points-to analysis is necessary to 
determine which object instances might exist, and then the result may still be inaccurate.

This is, however, not a problem with the \IEC family of languages~\cite{IEC61131:2003} used in this thesis. The \IEC 
standard defines languages for programmable logic controllers used in automation systems. Vendors implement the 
standard and provide tools, e.g.\ the \emph{Kemro IEC} implementation provided by our industry partner Keba AG. An 
important property of those languages is that they don't support dynamic allocation. Therefore, an object instance is 
statically allocated by defining a variable of a particular type. This makes it possible to statically determine all 
instances of objects that will exist at runtime, making it possible to build an instance-based SDG.


\section{The Tool} \label{sec:intro-tool}

Since SDGs can get very large---many hundreds of thousands of nodes 
\cite[sec.~4.3]{DBLP:conf/splc/AngererPLGG14}---using them for manual analysis in their entirety is impossible. There 
are, however, a number of use cases where one would want to explore the SDG of a system manually, for example, the 
aforementioned change impact analysis.

The tool developed in this thesis, named the \emph{\SB}, provides a means for interactive exploration of an SDG based 
on a chosen entry point. An entry point can be---depending on the use case---a single statement, a procedure, or a 
variable. Based on the selected entry point, control and data edges in the graph can be followed interactively. The 
tool was developed with a number of use cases in mind.

\begin{description}
  \item[Executions] Determine through which paths a statement or procedure can be executed. This effectively gives a 
  graph which contains all possible stack traces that could be observed for any execution.
  
  \item[Variable Assignment] Determine all statements that write to a particular variable (including procedure 
  parameters), and for those statements show their \emph{Executions}.
  
  \item[Change Impact] For any program element (statement, procedure, variable) determine other statements or variables 
  possibly affected by its change or execution. This can be used to determine whether a particular change will have any 
  unforeseen consequences.
  
  \item[Change Cause] This is basically an inverse change impact analysis. It determines for any program element
  other statements or variables which might have an impact on its value or execution.
\end{description}

The versions of the SDG and \SB introduced in this thesis work with the Kemro IEC languages only, because this was the 
initial target of Florian Angerer's PhD work. However there is ongoing work to support analysis across language 
boundaries as well~\cite{DBLP:conf/kbse/Angerer14}.


\section{Context} \label{sec:intro-context}

The \SB was developed in the course of our cooperation with \href{http://www.keba.com}{Keba AG} at the \emph{Christian 
Doppler Laboratory on Monitoring and Evolution of Very-Large-Scale Software Systems} (CDL MEVSS for short).

Keba AG is a company that makes tools, hardware, and software for the industrial automation domain. One of their major 
products is the KePlast platform, a solution for the automation of injection molding machines. At its core KePlast has 
a configurable control software framework implemented in Keba's proprietary dialect of the \IEC language (see also 
\cite[sec.~3.1]{DBLP:conf/splc/AngererPLGG14}).

Module 2 of the lab investigates development and evolution in large software ecosystems such as those developed by 
Keba.\footnote{See \url{http://mevss.jku.at/?page_id=654} for an overview of the motivation, goals, and results of CDL 
MEVSS Module 2.}
Key challenges in the development of such software systems are:

\begin{itemize}
  \item dealing with different variants and version of the software system;
  \item assessing the impact of changes to different product variants; and
  \item providing support to developers in evolving the system.
\end{itemize}

Our approach to these challenges is the FORCE environment~\cite{HinterreiterDA}, which provides feature-oriented and 
role-specific views of the software ecosystem. Feature-oriented views help in breaking the system into manageable 
pieces to work on, while role-specific views are useful to distinguish features managed by different teams.


\section{Outline}

The remainder of the thesis is organized as follows.
\autoref{ch:sdg} will cover the SDG in more detail and explain its structural properties.
In \autoref{ch:usecases} the use cases targeted by the \SB implementation will be introduced.
A tutorial and user's manual for the \SB, based on the use cases defined earlier, will be given in \autoref{ch:manual}.
\autoref{ch:impl} will explain some of the implementation details and serve as an introduction to the system for making
future extensions.
Finally, \autoref{ch:summary} will conclude the thesis.
