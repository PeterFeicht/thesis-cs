\chapter{Implementation} \label{ch:impl}

This chapter will describe the implementation of the system and will serve as an introduction for future developers.

The \SB is implemented as a plugin for the Eclipse workbench. It depends on several other MEVSS plugins which provide 
the AST classes and infrastructure for building the SDG, as well as an Eclipse editor for viewing IEC files. The \SB 
plugin provides two views (the \emph{\SB} view and the \emph{Instance Hierarchy} view) and contributes items to the IEC 
editor context menu.


\section{Overview}

This section will give an overview of the packages containing the \SB code and introduce important terminology along 
the way. All packages are part of a common root package \lstinline|at.jku.mevss.featureide.browserview|. The root 
package contains the \lstinline|DependencyBrowserPlugin| class, which represents a plugin instance and provides methods 
for showing AST and SDG nodes in the IEC editor. Also part of the root package is the 
\lstinline|DependencyBrowserController| class, which is the main entry point into the \SB, as well as the 
\lstinline|SDGBrowserSourceProvider| class, which makes the current model available across the workbench.

\begin{description}
  \lstitem{model2} This package contains the use cases which derive from the \lstinline|AbstractModel| base class. A 
  concrete use case is implemented by deriving from that base class and providing strategies for selecting nodes and 
  edges to be displayed. The \lstinline|AbstractModel| instance then builds a \emph{display graph} based on these 
  strategies.
  
  \lstitem{model2.graph} Classes in this package represent the \emph{display graph}, a model that abstracts from the 
  underlying SDG and allows the addition of container nodes (for representing hierarchy) and artificial graph elements. 
  This abstraction also allows the \SB to work with different SDG types without having to change the view 
  implementation. All nodes in the display graph can be in an \emph{expanded} or \emph{collapsed} state, meaning that 
  their logical successors will be shown or hidden, respectively. Furthermore, container nodes can be \emph{closed} so 
  their children are hidden to reduce visual clutter.
  
  All display graph elements derive from the common base class \lstinline|DisplayElement| which provides support for 
  attaching arbitrary properties. Classes \lstinline|DisplayNode| and \lstinline|DisplayEdge| represent nodes and 
  edges, respectively, and also store layout information (position, size or bend points). The \lstinline|DisplayGraph| 
  class stores all nodes and edges representing the graph and provides support for adding new elements. This class also 
  produces graph change events and provides a means for getting the visible part of the graph (elements 
  \emph{reachable} through expanded nodes).
  
  \lstitem{model2.hierarchy} The instance hierarchy is represented by the \lstinline|HierarchyTree| and 
  \lstinline|HierarchyTreeNode| classes in this package. Every \lstinline|AbstractModel| exposes an instance hierarchy 
  tree for its current graph contents.
  
  \lstitem{model2.layout} This package defines the classes used for the graph layout. The only implementation of 
  interface \lstinline|GraphLayouter| is \lstinline|KLayLayouter|, which uses the \emph{KLay Layered}\footnotemark{} 
  layout algorithm.
  
  \footnotetext{KLay Layered is a layer-based layout algorithm and part of the 
  \href{https://rtsys.informatik.uni-kiel.de/confluence/}{KIELER project}, it is currently being integrated into 
  Eclipse as part of the \href{http://www.eclipse.org/elk}{Eclipse Layout Kernel (ELK)}. For more information on KLay 
  Layered see \url{https://rtsys.informatik.uni-kiel.de/confluence/display/KIELER/KLay+Layered} and 
  \cite{DBLP:journals/vlc/SchulzeSH14}.}
  
  \lstitem{event} This package defines listener interfaces and event objects for several events: view changes (a new 
  model being displayed), selection of a node in the view, changes to a \lstinline|DisplayGraph|, and changes to 
  \lstinline|AbstractModel| properties.
  
  \lstitem{command} This package contains classes for interfacing an editor selection (path, source code range) with 
  the \SB controller, as well as the class \lstinline|SDGBrowserCommand|, which represents a command that operates on 
  display graph elements. Subpackage \lstinline|eclipse| defines Eclipse command handlers that plug into the IEC editor 
  context menu, subpackage \lstinline|pipe| defines handlers that receive commands via a Windows named pipe.
  
  \lstitem{command.feature} The interfaces in this package allow an arbitrary graph to be displayed in the \SB by 
  providing a \emph{feature} which may contain a number of \emph{slices} (sets of nodes to include in the graph). This 
  feature will be used by the FORCE platform~\cite{HinterreiterDA} to display a graph of those parts of the program 
  that belong to a certain feature.
  
  \lstitem{views} This package contains the views contributed to the Eclipse workbench. Class 
  \lstinline|InstanceHierarchyView| and supporting classes implement the \emph{Instance Hierarchy} view, which displays 
  the current model's hierarchy tree. The \lstinline|BrowserView| interface is used by the controller to render a graph 
  and is implemented by the abstract \lstinline|AbstractBrowserView| class. This class is an Eclipse view part that 
  provides several controls for setting model properties, but does not implement graph rendering itself.
  
  \lstitem{views.draw2d} This package and its subpackages contain the classes implementing the Draw2D\footnotemark{} 
  \SB view. The \lstinline|Draw2DBrowserView| class is an implementation of \lstinline|AbstractBrowserView| which uses 
  the \lstinline|GraphCanvas| class for rendering a display graph. The \lstinline|Resources| class exposes all the 
  colors and dimensions used for the displayed graph elements via static methods, so its implementation may be changed 
  in the future to allow Eclipse settings to be taken into account.
  
  \footnotetext{\href{https://www.eclipse.org/gef/draw2d/}{Draw2D} is a lightweight rendering toolkit on top of SWT, 
  the UI toolkit used by Eclipse.}
  
  The \lstinline|GraphCanvas| class implements most of the heavy lifting to support transitions when graph elements 
  enter or exit, implements zooming and hit testing, and handles events on displayed graph figures. The 
  \lstinline|animation| subpackage contains classes for supporting animations of Draw2D figures; the 
  \lstinline|overview| and \lstinline|tooltip| packages implement the scrollable overview thumbnail and tooltip 
  controls, respectively; all figures and decorations used for rendering are part of the \lstinline|figures| package.
\end{description}


\section{Display Graph}

TODO
